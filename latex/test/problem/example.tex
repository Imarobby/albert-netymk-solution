\documentclass[11pt]{article}
\usepackage{amsmath}
\usepackage{amssymb}
\usepackage{amsthm}
\usepackage{array}
\usepackage{xy}

\pdfpagewidth 8.5in
\pdfpageheight 11in
\topmargin -1in
\headheight 0in
\headsep 0in
\textheight 8.5in
\textwidth 6.5in
\oddsidemargin 0in
\evensidemargin 0in
\headheight 77pt
\headsep 0in
\footskip .75in
\setlength{\parindent}{0pt}

\begin{document}

\Large\underline{\textbf{Problem:}} \ Find all integer solutions to the equation
$$ \frac{1}{x} + \frac{1}{y} = \frac{1}{z} $$
\\
\\

\noindent  {\Large\underline{\textbf{Solution:}}}
\noindent \ We will show that all solutions are of the form:
\begin{eqnarray}
	x = k \cdot a \cdot \displaystyle \frac{a}{b} + \frac{c}{d} \\
	x = k \cdot a \cdot \displaystyle \frac{a}{b} + \frac{c}{d} \\
	x = k \cdot a \cdot \displaystyle (\frac{a}{b} + \frac{c}{d}) \\
	y = k \cdot b \cdot \left(a + b\right) \\
	z = k \cdot a \cdot b,
\end{eqnarray} 

\noindent where $k$, $a$, and $b$ are arbitrary non-zero integers, $a+b\neq0$. \\

\noindent  We can rewrite the given equation as
$$ z = \frac{xy}{x+y}, $$
so it suffices to find all pairs of integers $ x, y $ such that $x+y \mid xy$. \\

\noindent  \underline{\textbf{Lemma:}} \ Let $r$ and $s$ be relatively prime positive integers.  Then $r\pm s$ and $rs$ are relatively prime. \\

\noindent  \underline{\textbf{Proof:}} \ Suppose not; then there exists an integer $k > 1$ such that $k \mid r\pm s$ and $k \mid rs$.  We have \\  

\noindent  \[ k \mid r \pm s \Rightarrow k \mid r^2 \pm rs \Rightarrow k \mid r^2. \]
Similarly, $ k \mid s^2. $

Let $p > 1$ be any prime factor of $k$, so that $p \mid r^2$ and $p \mid s^2$.  
Now simply note that for any positive integer $m$, $ p \nmid m \Rightarrow p \nmid m^2$. 
Hence, we must have $p \mid r$ and $p \mid s$, which contradicts our assumption that $r$ and $s$ are relatively prime.  So $r \pm s$ and $rs$ must be relatively prime. // \\

Let $x$ and $y$ be integers such that $x+y \mid xy$.  We clearly cannot have $x+y = 0$.
If $\left|x+y\right| = 1$, then we have the solution sets
$$ \begin{array}{l}
	x = c \\
	y = -\left(c + 1\right) \\
	z = c\cdot\left(c+1\right)
\end{array} $$
and
$$ \begin{array}{l}
	x = c+1 \\
	y = -c \\
	z = -c\cdot\left(c+1\right),
\end{array} $$
where $c$ is an integer, $c \neq 0, 1$.

We now assume $\left|x+y\right| > 1 $.  
Let $n = \gcd\left(\left|x\right|,\left|y\right|\right)$.  Suppose that $\left|x\right|$ and $\left|y\right|$ are relatively prime.  
Then by the lemma, $\left|x+y\right|$ and $\left|xy\right|$ are relatively prime, a contradiction, since $x+y \mid xy$ and $\left|x+y\right| > 1 $.

\noindent Hence, $n > 1$.  Choose integers $a, b$ such that $x = na$ and $y = nb$.
This gives:
$$ \begin{array}{l}
	\ \  \ \left(na + nb\right) \mid \left(na\right) \cdot \left(nb\right) \\ 
	\Leftrightarrow n \cdot \left(a+b\right) \mid n^2 \cdot a \cdot b \\ 
	\Leftrightarrow a+b \mid n \cdot a \cdot b. 
\end{array} $$
Since $n = \gcd\left(\left|x\right|,\left|y\right|\right)$, $\left|a\right|$ and $\left|b\right|$ must be relatively prime.
So by the lemma, the above holds if and only if $a+b \mid n$.  Put $n = k \cdot \left(a + b\right)$.  This gives the solution set:

$$ \begin{array}{l}
	x = k \cdot a \cdot \left(a + b\right) \\
	y = k \cdot b \cdot \left(a + b\right) \\
	z = k \cdot a \cdot b.
\end{array} $$

\noindent Finally, note that the previous two solution sets are contained within this one (for the first set, take $k = -1, \ a = c, \ b = -\left(c+1\right)$; for the second set, take $k = 1, \ a = c+1, \ b= -c$).  Hence, this is the entire family of solutions, as desired.
\noindent   \hspace{\stretch{1}} $\blacksquare$ \\

\noindent \textbf{\textit{$\infty$ Michael Viscardi $\infty$}} \\

\noindent May 20, 2004

\end{document}
