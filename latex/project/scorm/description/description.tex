\documentclass[11pt]{report}
\begin{document}
\title{Project Description}
\author{Albert(Mingkun)}
\maketitle

\section{Introduction}
A Learning Management System(LMS) is a software application for the administration, documentation, tracking, and reporting of training programs, 
classroom and online events, e-learning programs, and training content.
\subsection{SCORM}
SCORM is a set of technical standards for e-learning software products. SCORM tells programmers how to write their code so that it can ``play well'' 
with other e-learning software. Specifically, SCORM governs how online learning content and Learning Management Systems (LMSs) communicate with each 
other.

\subsubsection{Version}
\begin{itemize}
	\item
		SCORM 2004

		4th Edition Released (March 31, 2009) — more stringent interoperability requirements, more flexible data persistence.
	\item
		SCORM Cloud

		The web is full of learning opportunity. But how to take advantage of it? How to connect what people do out on the web back to where you 
		normally track learning?

		You can reach learners where they are rather than waiting for them to come to your LMS.

		In addition to supplying best of class SCORM, SCORM Cloud serves as a connection point between your LMS and the rest of the internet.
\end{itemize}

\subsubsection{Overview}
SCORM specifiers that content should:
\begin{itemize}
	\item
		Be packaged in a ZIP file.
	\item
		Be described in an XML file.
	\item
		Communicate via JavaScript.
	\item
		Sequence using rules in XML.
\end{itemize}
SCORM is composed of three sub-specifications:
\begin{itemize}
	\item
		The Content Packaging section specifies how content should be packaged and described. It is based primarily on XML.
	\item
		The Run-Time section specifies how content should be launched and how it communicates with the LMS. It is based primarily on ECMAScript 
		(JavaScript).
	\item
		The Sequencing section specifies how the learner can navigate between parts of the course (SCOs). It is defined by a set of rules and 
		attributes written in XML.
\end{itemize}


\section{Task}
There are some LMSs, with some existing content. All of them are SCORM compliant. We want to import some information from other LMSs. Hopefully, we 
can send some information to other LMSs as well.
\subsection{Scenario}
Some companies have their own LMS, with some preexistent questions for their customers or employees to take. However, they do not have the appropriate 
grading system. Therefore, these companies want them to take the question in our non-SCORM LMS, with the questions created in their own LMS, possibly, 
combined with some questions in our LMS as well, then get the results. The results can be accessed by any SCORM compliant LMSs.
\section{Method}
It’s easy to put information on a website. It’s another thing to know whether anyone is learning that information. SCORM Cloud fills that gap. 
Unfortunately, in this case, it is not necessary to use it, for learner will go to the LMS to do the training. Ordinary SCORM suffices.

Various kinds of LMSs need to communicate with each other. SCORM defines the standards, so we need to make our LMS compliant, so that it can import 
and export SCORM compliant package. In this way, different LMSs can communicate with each other.
\end{document}
